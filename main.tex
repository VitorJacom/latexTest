\documentclass[12pt]{article}
\usepackage[utf8]{inputenc}
\usepackage[T1]{fontenc}
\usepackage[portuguese]{babel}
\usepackage{geometry}
\geometry{a4paper, margin=2.5cm}

\usepackage{graphicx}
\usepackage{xcolor}
\usepackage{hyperref}
\hypersetup{colorlinks=true, linkcolor=blue, urlcolor=purple}

\usepackage{amsmath, amssymb}
\usepackage{booktabs}

\usepackage{listings}
\lstset{inputencoding=utf8}        % ← permite acentos nos trechos de código

\usepackage{tikz}
\usepackage{pgfplots}
\pgfplotsset{compat=1.18}

\usepackage[ruled,vlined]{algorithm2e}
\usepackage{multicol}
\usepackage{enumitem}

\usepackage{imakeidx}
\makeindex

\usepackage[acronym]{glossaries}
\makeglossaries

\usepackage{csquotes}              % ← recomendado pelo biblatex
\usepackage{biblatex}
\addbibresource{references.bib}

% Glossário e acrônimos
\newglossaryentry{latex}{
  name=LaTeX,
  description={sistema de preparação de documentos}
}
\newacronym{pdf}{PDF}{Portable Document Format}

\title{Tutorial Completo de Recursos do \\ \LaTeX\ para Geração de PDF}
\author{Vitor Jacom}
\date{\today}

\begin{document}

\maketitle
\tableofcontents
\newpage

\section{Introdução}
O objetivo deste documento é demonstrar os principais recursos do \gls{latex} para produção de documentos PDF profissionais, prontos para impressão ou distribuição digital. Para citar o formato final usamos o acrônimo Teste \gls{pdf}.

\section{Formatação de Texto}\label{sec:texto}
\index{Texto!\LaTeX}
Podemos \textbf{destacar} informações em \textit{itálico}, \underline{sublinhado} ou
\texttt{monoespaçado}. Além disso, listas são simples de criar:

\begin{itemize}[noitemsep]
  \item Primeiro item
  \item Segundo item
\end{itemize}

Listas numeradas:

\begin{enumerate}[label=\arabic*.]
  \item Passo 1
  \item Passo 2
\end{enumerate}

\section{Referências Internas e Links}
Podemos referenciar a Seção~\ref{sec:texto} ou clicar no
\hyperref[sec:matematica]{exemplo de matemática}. Links externos também funcionam: \url{https://www.latex-project.org}.

\section{Imagens}
A Figura~\ref{fig:exemplo} ilustra a inclusão de arquivos gráficos.

\begin{figure}[h!]
\centering
\includegraphics[width=0.5\textwidth]{figures/default.jpg}
\caption{Imagem de exemplo}\label{fig:exemplo}
\end{figure}

\section{Tabelas}
\begin{table}[h!]
\centering
\begin{tabular}{lcr}
\toprule
\textbf{Nome} & \textbf{Cargo} & \textbf{Idade} \\
\midrule
Alice & Engenheira & 29 \\
Bob   & Analista   & 35 \\
\bottomrule
\end{tabular}
\caption{Tabela de exemplo}\label{tab:exemplo}
\end{table}

\section{Matemática}\label{sec:matematica}
Equação inline $E = mc^2$ e bloco:
\[
\int_{0}^{1} x^2 \, dx = \frac{1}{3}
\]
Matrizes:
\[
A = \begin{bmatrix}
1 & 2 \\
3 & 4
\end{bmatrix}
\]

\section{Gráficos com TikZ/PGFPlots}
\begin{tikzpicture}
  \begin{axis}[width=8cm, height=6cm, xlabel={$x$}, ylabel={$y=x^2$}]
    \addplot[color=blue, domain=0:4]{x^2};
  \end{axis}
\end{tikzpicture}

\section{Código Fonte}
\begin{lstlisting}[language=Python, caption=Exemplo em Python]
def hello():
    print("Ola, LaTeX!")
\end{lstlisting}

\section{Algoritmos}
\begin{algorithm}[H]
\SetAlgoLined
\KwIn{Lista $L$ de números}
\KwOut{Lista ordenada}
\Repeat{lista ordenada}{
  percorra $L$ comparando elementos adjacentes\;
  troque se estiverem fora de ordem\;
}
\caption{Bubble Sort simplificado}
\end{algorithm}

\section{Multicolunas}
\begin{multicols}{2}
Este parágrafo aparece em duas colunas, útil para relatórios ou artigos.
\end{multicols}

\section{Glossário e Acrônimos}
A palavra \gls{latex} e o acrônimo \gls{pdf} foram definidos no preâmbulo.

\printglossary[type=\acronymtype]
\printglossary

\section{Bibliografia}
Exemplo de citação~\cite{knuth1984texbook}.

\printbibliography

\clearpage
\printindex

\end{document}
